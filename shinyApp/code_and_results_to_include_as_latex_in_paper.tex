\documentclass{article}

\usepackage{Sweave}
\begin{document}
\input{code_and_results_to_include_as_latex_in_paper-concordance}

%code_and_results_to_include_in_latex_paper
\begin{Schunk}
\begin{Sinput}
> library(Brec)
> library(plotly)
> library(ggplot2)
> library(reshape2)
> library(shiny)
> library(na.tools)
> library(tidyverse)
> library(xtable)
> plots_path =  paste0(getwd(),"/../DQC/")
> ## === Read data and build whole chromosomes from arms
> 
> # load inputs data
> # inputData = read.csv(file = "data/Dmel_R6_formatted_v2.csv", header = T, sep = "\t")
> inputData = read.csv(file = "../data/BREC_byArm_MB_marey_map_R5.36_removeOutliers.csv", header = T, sep = "\t")
> chrList = get_list_of_chromosomes(inputData)
> newChrList = c("X", "2L", "2R", "3L", "3R")
\end{Sinput}
\end{Schunk}

\begin{Schunk}
\begin{Sinput}
> options(xtable.floating = FALSE)
> options(xtable.timestamp = "")
> xtab <- xtable(head(inputData))
> print(xtab)
\end{Sinput}
\begin{Soutput}
% latex table generated in R 3.5.2 by xtable 1.8-4 package
% 
\begin{tabular}{rllrr}
  \hline
 & mkr & chr & cm & mb \\ 
  \hline
1 & l(2)gl & 2L & 0.00 & 0.02 \\ 
  2 & net & 2L & 0.00 & 0.08 \\ 
  3 & Sam-S & 2L & 0.00 & 0.11 \\ 
  4 & Gs1 & 2L & 0.05 & 0.13 \\ 
  5 & kis & 2L & 0.00 & 0.23 \\ 
  6 & smo & 2L & 0.40 & 0.28 \\ 
   \hline
\end{tabular}
\end{Soutput}
\end{Schunk}

\section{Validation}

\subsection{Chromatin boundaries validation table}

\begin{Schunk}
\begin{Sinput}
> cyto_ref_hcb = data.frame()
> hcb_validation_df = data.frame()
> 
> 
\end{Sinput}
\end{Schunk}

\subsection{Chromatin boundaries validation plot(ly)s}
\begin{Schunk}
\begin{Sinput}
> 
\end{Sinput}
\end{Schunk}


\subsection{Chromatin boundaries validation shift ggplot}
\begin{Schunk}
\begin{Sinput}
> 
> 
\end{Sinput}
\end{Schunk}

\begin{Schunk}
\begin{Sinput}
> 
\end{Sinput}
\end{Schunk}

\section{Validation of recombination rates with the high resolution map from FlyBase}

\begin{Schunk}
\begin{Sinput}
> # Load and prepare Langley data (with plotting the content)
> langleyData <- read.csv(file = "../datasets/Langley_TableS12.txt", skip= 3, sep="\t", header=TRUE)
>     print(head(langleyData))
\end{Sinput}
\begin{Soutput}
  Cytogenetic  cM     end      M.bp  X Cytogenetic.1 cM.1   end.1   M.bp.1 X.1
1          1A 0.0  302615 -2.19e-09 NA           21A  0.0   22221 8.37e-09  NA
2          1B 0.0  554516 -1.29e-09 NA           21B  0.5  324485 1.49e-08  NA
3          1C 0.0  725129  5.04e-10 NA           21C  0.7  464653 1.88e-08  NA
4          1D 0.0  885080  2.03e-09 NA           21D  1.0  518418 2.04e-08  NA
5          1E 0.1 1148842  4.22e-09 NA           21E  1.5 1086654 2.58e-08  NA
6          1F 0.1 1262585  6.27e-09 NA           21F  2.0 1326937 3.21e-08  NA
  Cytogenetic.2 cM.2   end.2    M.bp.2 X.2 Cytogenetic.3 cM.3   end.3   M.bp.3
1           41A   55  469718  3.61e-10  NA           61A -1.0  114102 2.41e-09
2           41B   55  592171 -1.34e-11  NA           61B -1.0  298520 2.63e-09
3           41C   55  890835 -1.91e-10  NA           61C -0.5  756302 2.53e-09
4           41D   55 1034083 -3.34e-10  NA           61D -0.5  924623 3.90e-09
5           41E   55 1293315 -4.65e-10  NA           61E -0.5 1082762 5.17e-09
6           41F   55 1690404 -6.04e-10  NA           61F  0.0 1393741 7.75e-09
  X.3 Cytogenetic.4 cM.4   end.4   M.bp.4
1  NA           82A 47.1  236357 1.41e-10
2  NA           82B 47.1  326924 4.46e-10
3  NA           82C 47.1  482821 8.33e-10
4  NA           82D 47.1  689062 1.31e-09
5  NA           82E 47.1  926850 1.68e-09
6  NA           82F 47.5 1189877 1.90e-09
\end{Soutput}
\begin{Sinput}
>     langleyData_X = langleyData[ , c(1:4)]
>     langleyData_2L = langleyData[ , c(6:9)]
>     langleyData_2R = langleyData[ , c(11:14)]
>     langleyData_3L = langleyData[ , c(16:19)]
>     langleyData_3R = langleyData[ , c(21:24)]
>     newChrList = c("X", "2L", "2R", "3L", "3R")
>     for (chrID in newChrList) {
+ 
+         # chrID = "3R"
+         if(chrID == "X"){
+             langleyData = langleyData_X
+         }else if(chrID == "2L"){
+             langleyData = langleyData_2L
+         }else if(chrID == "2R"){
+             langleyData = langleyData_2R
+         }else if(chrID == "3L"){
+             langleyData = langleyData_3L
+         }else if(chrID == "3R"){
+             langleyData = langleyData_3R
+         }
+ 
+         names(langleyData)[c(1:4)] = c("Cytogenetic", "cM", "end", "M.bp")
+ 
+         # ------------------plotting with ggplot ==> second y axis issues ==> solved with plotly -------------
+         # p = ggplot(langleyData, aes(x = end/10^6)) + geom_point(aes( y = cM, color = "Genetic markers"))
+         #
+         # # adding regression line
+         # p = p + geom_line(aes(y = M.bp*10^9, color = "Recombination rate"))
+         #
+         # # now adding the secondary axis, following the example in the help file ?scale_y_continuous
+         # # and, very important, reverting the above transformation
+         # ymin = min(langleyData$cM, langleyData$M.bp*10^9, na.rm = T)
+         # ymax = max(langleyData$cM, langleyData$M.bp*10^9, na.rm = T)
+         # p = p + scale_y_continuous(sec.axis = sec_axis(~./10 , name = "Recombination rate (cM/Mb)"))#, limits = c(-1, 8))
+         #
+         # # modifying colours and theme options
+         # p = p + scale_colour_manual(values = c("black", "red"))
+         # p = p + labs(y = "Genetic distance (cM)",
+         #               x = "Physical location (Mb)",
+         #               colour = paste(chrID, " - Langley data"))
+         # p = p + theme(legend.position = c(0.2, 0.9))
+         # print(p)
+         # ggsave(filename = paste0("chr_",chrID,"_Langley_", ".png") #
+         #        , plot = p
+         #        , path = plots_path #plot = last_plot()
+         #        , height = 8
+         #        , width = 12
+         # )
+ 
+         # ------------------plotting with plotly ----------------------------------------------------------------
+ 
+          pLangley = plot_ly(data = langleyData, x = ~end/10^6, y= ~cM,
+                   type = 'scatter', mode = 'markers',
+                   yaxis = "y", name = "genetic markers"
+                   # hoverinfo = text,
+                   # text = ~paste('Mb:', x, "\ncM:",y)
+                   ) %>%  # plot dataSet
+             add_trace(data = langleyData, x= ~end/10^6, y = ~M.bp*10^8,  #polynomial
+                       type = "scatter", mode = "lines",
+                       yaxis = "y2", name = "RR estimates",
+                       line = list(color = "red"))  %>%
+ 
+              layout(title = paste("Chromosome", chrID),
+                    xaxis = list(
+                      showline = TRUE
+                      ,title = paste( "Physical location (Mb)" )
+                      ,titlefont = list(size= 20)
+                      ,color = "#0072B2"
+                      ,zeroline = FALSE
+                      ,showgrid = FALSE
+                      ,exponentformat = "power"
+                      ,tickwidth = 3
+                      ,tickcolor = "#0072B2"
+                      ,showticklabels = TRUE
+                      ,tickfont = list(size = 18)
+                    )
+                    ,yaxis = list(  # the data points axis is invisible since it matches the one of the polynomial
+                        showline = TRUE
+                        ,side = "left"
+                        ,title = "Genetic location (cM)"
+                        ,titlefont = list(size= 20)
+                        ,color = "#0072B2"
+                        ,zeroline = FALSE
+                        ,showgrid = FALSE
+                        ,exponenformat = "power"
+                        ,tickwidth = 3
+                        ,tickcolor = "#0072B2"
+                        ,showticklabels = TRUE
+                        ,tickfont = list(size = 18)
+                    )
+                    ,yaxis2 = list(
+                        showline = TRUE
+                      ,side = "right"
+                      ,overlaying = "y"
+                      ,title = "Recombination rate (cM/Mb)"
+                      ,titlefont = list(size= 20)
+                      ,color = "red" #2CA02C"
+                      ,zeroline = FALSE
+                      ,anchor = "y"
+                      ,showgrid = FALSE
+                      ,exponentformat = "power"
+                      ,tickwidth = 3
+                      ,showticklabels = TRUE
+                      ,tickfont = list(size = 18)
+                    )
+          )
+          print(pLangley)
+ 
+ 
+     }
\end{Sinput}
\end{Schunk}

\section{Comparison of regression models for RR estimates}

\paragraphe{Caption}Comparison of multiple regression models for recombination rate estimates with the high resolution recombination map from FlyBase, along \textit{D. melanogaster} R5 chromosomes. Shown here are third degree polynomial and Loess with span values of 15\%, 25\%, 0.5\% and 0.75\%.

\begin{Schunk}
\begin{Sinput}
> 
\end{Sinput}
\end{Schunk}

\section{DQC}
\subsection{Density}

\begin{Schunk}
\begin{Sinput}
> densityList= list()
> for (chrID in chrList[-c(1,2)]) {
+ 
+     ## R5_assign ref cyto boundaries according to chr in process
+     if(chrID == "2L"){
+         c = 19.954780
+         t = 0.698949
+     }else if(chrID == "2R"){
+         c = 6.090470
+         t = 20.020890
+     }else if(chrID == "3L"){
+         c = 18.408033
+         t = 0.356604
+     }else if(chrID == "3R"){
+         c = 8.349278
+         t = 27.248244
+     }else if(chrID == "X"){
+         c =  20.665672
+         t =  2.460008
+     }
+     print(c("=========== chr in process ====> ", chrID, "==================="))
+     refChromosome = get_chromosome_from_inputData(inputData, chrID)
+     print(c("refChromosomeSize",nrow(refChromosome)))
+     # cleaning step
+     refChromosome = clean_chromosome_data(refChromosome, genomeName = "DmelR6", chrID) #=> chr2 : 240 instead of 267
+     refChromosomeSize = nrow(refChromosome)
+     print(c("refChromosome after cleaning --> size = ", refChromosomeSize))
+ 
+     #prepare refchr for DQC
+     # refChromosomeAsMatrix = as.matrix.data.frame(subset(refChromosome, select=c("cm", "mb")), rownames.force = NA)
+ 
+     # print(c("frac = ", frac, "n = ", nrow(testChromosome)))
+     # ------------- simu1List : number of markers per chromosome ---------------------------------
+     # enoughData = FALSE
+     # simu1List = list()  # simulated chromosomes baesd on number of markers
+     # simu1ListOfPvals = c()
+ 
+   tictoc::tic()
+ 
+     testChromosome = refChromosome
+     # decacl = c() # centro_left decalage_of_Brec_estimated_HCB_compared_To_ref_cyto_HCB
+     # decacr = c() # centro_right
+     # decatl = c() # telo_left
+     # decatr = c() # telo_right
+ 
+     deca_c = c() # centro decalage_of_Brec_estimated_HCB_compared_To_ref_cyto_HCB
+     deca_t = c() # telo
+ 
+     mkrDensity = c()
+     densityFD = data.frame()
+ 
+     testSizes = seq(1, 0.30, by = -0.05) # simulation fractions vector
+ 
+     for(frac in testSizes) {#testSizes each 5%
+         # hcb_cl = c()
+         # hcb_cr = c()
+         # hcb_tl = c()
+         # hcb_tr = c()
+ 
+         hcb_c = c()
+         hcb_t = c()
+ 
+         if (frac != 1) { ## not the case of 100%
+             for (i in c(1:30)) {
+                 print(c("Test for : frac =", frac," No ", i, "  ---------------------------------"))
+                 testChromosome = dplyr::sample_frac(refChromosome, frac)
+                 testChromosome = testChromosome[order(testChromosome$mb),]  #sort by ascendaing mb which is always true (logically)
+ 
+                 # simu1List = rlist::list.append(simu1List, testChromosome)
+                 # chi2 = chisq.test(refChromosomeAsMatrix, p = testChromosome, simulate.p.value = TRUE, rescale.p = TRUE) # to rescale p to sum up to 1
+                 # simu1ListOfPvals = c(simu1ListOfPvals, chi2$p.value)
+ 
+                 gg1 = ggplot(data = testChromosome, mapping = aes(x=mb, y=cm)) + geom_point()
+                 ggsave(filename = paste0("chr_",chrID, "_SimuChr-NbrMrks=", frac*100, "percent_testNo_",i,".png"), plot = gg1, path = plots_path) #plot = last_plot()
+ 
+                 # run BREC to get new HCB
+                 RR_object = estimate_recombination_rates_loess(testChromosome, spanVal = 0.15)
+                 print("RR done")
+ 
+                 minRR_object = get_min_RR_value_based_on_polynomial(testChromosome)
+                 print(" get minRR done !")
+                 print(minRR_object)
+ 
+                 chrType_object = get_chromosome_type(testChromosome, chrID, minRR_object, RR_object)
+                 chrType = chrType_object$chr_type
+                 print(chrType_object)
+ 
+                 R2DataFrame2D = compute_cumulated_R_squared_2directions(testChromosome)
+                 print("R2 done")
+ 
+                 # if(using_slidingWindowApproach_for_HCB) {
+                 print("Extracting CB for this chromosome ...")
+                 heteroChromatinBoundaries = extract_CB(testChromosome, RR_object, R2DataFrame2D, chrID, chrType, minRR_object)
+                 print("extract centroCB done")
+                 telomeres_boundaries = extract_telomeres_boundaries(testChromosome, R2DataFrame2D, chrID, chrType, minRR_object)
+                 print("extract telo CB done")
+                 RR_object = extrapolate_RR_estimates(testChromosome, RR_object, heteroChromatinBoundaries, telomeres_boundaries, chrID, chrType, minRR_object)
+                 print("extrapolation done")
+ 
+             #     hcb_cl = c(hcb_cl, heteroChromatinBoundaries$heteroBoundLeft)
+             #     hcb_cr = c(hcb_cr, heteroChromatinBoundaries$heteroBoundRight)
+             #     hcb_tl = c(hcb_tl, telomeres_boundaries$extrapolPhysPos_left)
+             #     hcb_tr = c(hcb_tr, telomeres_boundaries$extrapolPhysPos_right)
+ 
+                 hcb_c = c(hcb_c, heteroChromatinBoundaries$heteroBoundArm)
+                 hcb_t = c(hcb_t, telomeres_boundaries$telo_arm)
+             }
+ 
+         }else{ ## case of  100% --> run only once not 30 times since it's the same
+             print("Test for : original_cleaned_chromosome ---------------------------------")
+             # simu1List = rlist::list.append(simu1List, testChromosome)
+             # chi2 = chisq.test(refChromosomeAsMatrix, p = testChromosome, simulate.p.value = TRUE, rescale.p = TRUE) # to rescale p to sum up to 1
+             # simu1ListOfPvals = c(simu1ListOfPvals, chi2$p.value)
+ 
+             gg2 = ggplot(data = testChromosome, mapping = aes(x=mb, y=cm)) + geom_point()
+             ggsave(filename = paste0("chr_",chrID, "_SimuChr-NbrMrks=", frac*100, "percent_original_cleaned_chromosome.png"), plot = gg2, path = plots_path)
+ 
+             # run BREC to get new HCB
+             RR_object = estimate_recombination_rates(testChromosome)
+             print("RR done")
+ 
+             RR_object = estimate_recombination_rates_loess(testChromosome, spanVal = 0.15)
+             print("RR done")
+ 
+             minRR_object = get_min_RR_value_based_on_polynomial(testChromosome)
+             print(" get minRR done !")
+             print(minRR_object)
+ 
+             chrType_object = get_chromosome_type(testChromosome, chrID, minRR_object, RR_object)
+             chrType = chrType_object$chr_type
+             print(chrType_object)
+ 
+             R2DataFrame2D = compute_cumulated_R_squared_2directions(testChromosome)
+             print("R2 done")
+ 
+             # if(using_slidingWindowApproach_for_HCB) {
+             print("Extracting CB for this chromosome ...")
+             heteroChromatinBoundaries = extract_CB(testChromosome, RR_object, R2DataFrame2D, chrID, chrType, minRR_object)
+             print("extract centroCB done")
+             telomeres_boundaries = extract_telomeres_boundaries(testChromosome, R2DataFrame2D, chrID, chrType, minRR_object)
+             print("extract telo CB done")
+             RR_object = extrapolate_RR_estimates(testChromosome, RR_object, heteroChromatinBoundaries, telomeres_boundaries, chrID, chrType, minRR_object)
+             print("extrapolation done")
+ 
+             # hcb_cl = c(hcb_cl, heteroChromatinBoundaries$heteroBoundLeft)
+             # hcb_cr = c(hcb_cr, heteroChromatinBoundaries$heteroBoundRight)
+             # hcb_tl = c(hcb_tl, telomeres_boundaries$extrapolPhysPos_left)
+             # hcb_tr = c(hcb_tr, telomeres_boundaries$extrapolPhysPos_right)
+ 
+             hcb_c = c(hcb_c, heteroChromatinBoundaries$heteroBoundArm)
+             hcb_t = c(hcb_t, telomeres_boundaries$telo_arm)
+ 
+             RR_plot_100pc = plot_all(testChromosome, RR_object, genomeName ="DmelR6_DQC1_SimuDmelR6Chr2_100pc", toString(chrID), R2DataFrame2D, heteroChromatinBoundaries, heteroChromatinBoundaries$swSize, telomeres_boundaries, plots_path, chrType)
+ 
+             # save_plot_as_png(RR_plot_100pc, plots_path, genomeName = "DmelR6", chrID)
+             print(RR_plot_100pc)
+         }
+ 
+         print(c("frac = ", frac, "n = ", nrow(testChromosome)))
+         mkrDensity = c(mkrDensity, nrow(testChromosome))
+ 
+         # mhcb_cl = mean(abs(cl - hcb_cl))
+         # mhcb_cr = mean(abs(cr - hcb_cr))
+         # mhcb_tl = mean(abs(tl - hcb_tl))
+         # mhcb_tr = mean(abs(tr - hcb_tr))
+         #
+         # decacl = c(decacl, mhcb_cl)
+         # decacr = c(decacr, mhcb_cr)
+         # decatl = c(decatl, mhcb_tl)
+         # decatr = c(decatr, mhcb_tr)
+ 
+         mhcb_c = mean(abs(c - hcb_c))
+         mhcb_t = mean(abs(t - hcb_t))
+ 
+         deca_c = c(deca_c, mhcb_c)
+         deca_t = c(deca_t, mhcb_t)
+ 
+     }
+ 
+     mkrDensity
+     densityList = rlist::list.append(densityList, mkrDensity )
+ 
+     fractions = testSizes*100  #[1:10]
+     # res = data.frame(fractions, telo_left = decatl, centro_left = decacl, centro_right = decacr, telo_right = decatr)
+     res = data.frame(fractions, telo = deca_t, centro = deca_c)
+     d <- melt(res, id="fractions")
+ 
+     gg3 = ggplot(d, aes(x=fractions, y = value, color=variable)) +
+         geom_line(aes(linetype=variable), size=1) +
+         geom_point(aes(shape=variable, size=4)) +
+         scale_linetype_manual(values = c(2,1,1,2)) +
+         scale_shape_manual(values= c(15,17,17,15))
+     ggsave(filename = paste0("chr_",chrID,  "_SimuResuts_1.png"), plot = gg3, path = plots_path)
+ 
+ tictoc::toc()
+ }